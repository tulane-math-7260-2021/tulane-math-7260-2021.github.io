\documentclass[12pt]{article} % use larger type; default would be 10pt
\RequirePackage[colorlinks,citecolor=blue,urlcolor=blue]{hyperref}
\usepackage[utf8]{inputenc} % set input encoding (not needed with XeLaTeX)

\usepackage{geometry} % to change the page dimensions
\geometry{letterpaper} % or letterpaper (US) or a5paper or....
\geometry{margin=1in} % for example, change the margins to 2 inches all round

\usepackage{graphicx} % support the \includegraphics command and options
\usepackage{amsbsy}
\usepackage{amsmath}
\usepackage{amssymb}
\usepackage{amsfonts}
\usepackage{cite}
\usepackage{float}
\usepackage[toc,page]{appendix}
\newcommand{\vecc}[1]{\mathbf{#1}}
%\toggletrue{showProof}
\begin{document}

\title{Time-dependent rate model}

\maketitle

\subsection*{The clock model in the partial vectors for one branch}
The partial differential equation
$$
\frac{d \vecc{P}(t)}{dt} = \vecc{P}(t) \vecc{Q} \mu(t)
$$
has the solution involving matrix exponential for one branch of time $(t_1, t_2)$:
$$
\log \vecc{P}(t_2) - \log \vecc{P}(t_1) = \int\limits_{t_1}^{t_2} \vecc{Q}\mu(t)dt
$$
\begin{enumerate}
	\item For constant $\mu(t) = r$.
	$$
	\vecc{P}(t_2) = \vecc{P}(t_1) e^{\vecc{Q}r(t_2 - t_1)}
	$$
	that translates to
	$$
	r(t_1, t_2) = r
	$$
	\item For a linear (i.e.~time dependent) function, $\mu(t) = \beta t$
	$$
	\begin{aligned}
		\vecc{P}(t_2) &= \vecc{P}(t_1) e^{\frac{1}{2} \vecc{Q}\beta(t_2^2 - t_1^2)}\\
		&= \vecc{P}(t_1) e^{ \vecc{Q}(t_2 - t_1) \beta \frac{t_1 + t_2}{2}}\\
	\end{aligned}
	$$
	which suggests a mid-point for node heights with
	$$
	r(t_1, t_2) = \beta \frac{t_1 + t_2}{2}
	$$
	\item For a linear function with a upper threshold,
	$$
	\mu(t) = \left\{ \begin{array}{cc}
		\beta t & t \ge T\\
		\beta T & t < T\\
	\end{array} \right.
   $$
   has 
   $$
   \vecc{P}(t_2) = \left\{ \begin{array}{cc}
   	\vecc{P}(t_1)e^{\vecc{Q}\beta T (t_2 - t_1)} & t_1 < t_2 < T\\
   	\vecc{P}(t_1)e^{\vecc{Q}\beta T (T - t_1) + \vecc{Q}(t_2 - T)\beta \frac{t_2 + T}{2}} & t_1 < T < t_2\\
   	\vecc{P}(t_1)e^{\vecc{Q} (t_2 - t_1) \beta \frac{t_1 + t_2}{2}} & T < t_1 < t_2 \\
   \end{array} \right.
   $$
   that translates into
   $$
   \vecc{r}(t_1, t_2) = \left\{ \begin{array}{cc}
   	\beta T & t_1 < t_2 < T\\
   	\beta \frac{\frac{1}{2}(t_2^2 - T^2) + T(T - t_1)}{t_2 - t_1} & t_1 < T < t_2\\
   	\beta \frac{t_1 + t_2}{2} & T < t_1 < t_2 \\
   	\end{array} \right.
   $$
\end{enumerate}


\end{document}
















