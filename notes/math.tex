\RequirePackage[colorlinks,citecolor=blue,urlcolor=blue]{hyperref}
\usepackage[utf8]{inputenc} % set input encoding (not needed with XeLaTeX)

%%% Examples of Article customizations
% These packages are optional, depending whether you want the features they provide.
% See the LaTeX Companion or other references for full information.

%%% PAGE DIMENSIONS
\usepackage{geometry} % to change the page dimensions
\geometry{letterpaper} % or letterpaper (US) or a5paper or....
\geometry{margin=1in} % for example, change the margins to 2 inches all round
% \geometry{landscape} % set up the page for landscape
%   read geometry.pdf for detailed page layout information

\usepackage{graphicx} % support the \includegraphics command and options
%\usepackage{setspace}

% \usepackage[parfill]{parskip} % Activate to begin paragraphs with an empty line rather than an indent

%%% PACKAGES
\usepackage{booktabs} % for much better looking tables
\usepackage{array} % for better arrays (eg matrices) in maths
\usepackage{paralist} % very flexible & customisable lists (eg. enumerate/itemize, etc.)
\usepackage{verbatim} % adds environment for commenting out blocks of text & for better verbatim
\usepackage{subfig} % make it possible to include more than one captioned figure/table in a single float
\usepackage{amsbsy}
\usepackage{amsmath}
\usepackage{amssymb}
\usepackage{amsfonts}
\usepackage{amscd}
\usepackage{amsthm}
\usepackage{mathabx}
\usepackage{cite}
\usepackage{dutchcal}
\usepackage{float}
\usepackage[toc,page]{appendix}
%\usepackage{natbib}
%\usepackage{lineno}
\usepackage{mathrsfs}
%\usepackage[nolists]{endfloat}
\usepackage{todonotes}
\usepackage{setspace}
\usepackage{multirow}
\usepackage{algorithm}
\usepackage{algorithmicx}
\usepackage{algpseudocode}
\usepackage{listings}
\usepackage{parskip}
%\usepackage{mathptmx}
%\DeclareMathAlphabet{\mathcal}{OMS}{cmsy}{m}{n}
% These packages are all incorporated in the memoir class to one degree or another...

%%% HEADERS & FOOTERS
\usepackage{fancyhdr} % This should be set AFTER setting up the page geometry
\pagestyle{fancy} % options: empty , plain , fancy
\renewcommand{\headrulewidth}{0pt} % customise the layout...
\lhead{}\chead{}\rhead{}
\lfoot{}\cfoot{\thepage}\rfoot{}

%%% SECTION TITLE APPEARANCE
\usepackage{sectsty}
\allsectionsfont{\sffamily\mdseries\upshape} % (See the fntguide.pdf for font help)
% (This matches ConTeXt defaults)

%%% ToC (table of contents) APPEARANCE
\usepackage[nottoc,notlof,notlot]{tocbibind} % Put the bibliography in the ToC
\usepackage[titles,subfigure]{tocloft} % Alter the style of the Table of Contents
\renewcommand{\cftsecfont}{\rmfamily\mdseries\upshape}
\renewcommand{\cftsecpagefont}{\rmfamily\mdseries\upshape} % No bold!

%%% END Article customizations

%%% The "real" document content comes below...
%Variable Declarations
%Matricies
%\graphicspath{{Figures/}}

\def\diffvar{\boldsymbol{\Sigma}}

\newcommand{\vecEpsilon}[0]{\boldsymbol{\epsilon}}
\newcommand{\Expected}[1]{{\mathbf{E}\left(#1\right)}}
\newcommand{\Var}[1]{\mathbf{Var}\left(#1\right)}
\newcommand{\vecc}[1]{\mathbf{#1}}
\newcommand{\p}[1]{{\mathbf{p}}_{#1}}
\newcommand{\q}[1]{{\mathbf{q}}_{#1}}
\newcommand{\M}[1]{{\mathbf{M}}_{#1}}
\newcommand{\Q}[1]{{\mathbf{Q}}_{#1}}
\newcommand{\Smat}[1]{{\mathbf{S}}_{#1}}
\newcommand{\X}[1]{{\mathbf{X}_{#1}}}
\newcommand{\D}[1]{{\mathbf{D}_{#1}}}
\newcommand{\V}[1]{{\mathbf{V}_{#1}}}
\newcommand{\G}[1]{{\mathbf{G}_{#1}}}
\newcommand{\bl}[1]{{b}_{#1}} % for branch length
\newcommand{\br}[1]{{r}_{#1}} % for branch rate
\newcommand{\bp}[1]{{\nu}_{#1}} % for branch parameter
\newcommand{\sr}[1]{{\gamma}_{#1}} % for site rate
\newcommand{\Ptr}[1]{\mathbf{P}_{#1}}

\newcommand{\cDensity}[2]{\ensuremath{P(#1 \,|\,#2)}}
\newcommand{\veccDensity}[2]{\mathbf{p}(#1 \,|\,#2)}
\newcommand{\matcDensity}[2]{\mathbf{P}(#1 \,|\,#2)}
\newcommand{\secondDerivative}[3]{\frac{\partial ^2}{\partial #1 _i\partial #2 _j}{#3}}
\newcommand{\density}[1]{\ensuremath{p(#1 )}}
\newcommand{\vecOperator}[1]{\mbox{vec}\hspace{-0.1em}\left[ #1 \right]}
\newcommand{\trOperator}[1]{\mbox{tr}\hspace{-0.1em}\left[ #1 \right]}
\newcommand{\oneVector}[1]{\mathbf{1}_{#1}}
\newcommand{\order}[1]{{\cal O}\hspace{-0.1em}\left( #1 \right)}
\newcommand{\specialInverse}{-}
\newcommand{\inverse}{^{-1}}
\newcommand{\transpose}{^T}
\newcommand{\vectorize}[1]{\text{vec}\left(#1\right)}
\newcommand{\kronecker}{\otimes}

\newcommand{\parent}[1]{\text{pa}(#1)}
\newcommand{\parentNth}[2]{\text{pa}^{(#2)}(#1)}
\newcommand{\sibling}[1]{\text{sib}(#1)}
\newcommand{\smallParent}[1]{\text{\tiny pa}(#1)}
\newcommand{\nodePrecision}[1]{\mathbf{P}_{#1}}
\newcommand{\nodeVariance}[1]{\mathbf{\Sigma}_{#1}}
\newcommand{\nodeVarianceFunction}[1]{\nodeVariance{}\hspace{-0.1em}\left( #1 \right)}
\newcommand{\deflatedNodePrecision}[1]{\nodePrecision{#1}^{\star}}
\newcommand{\nodeMean}[1]{\mathbf{m}_{#1}}
\newcommand{\nodeIndexOne}{i}
\newcommand{\nodeIndexTwo}{j}
\newcommand{\nodeIndexThree}{k}
\newcommand{\branchLength}[1]{t_{#1}}
\newcommand{\diffusionVariance}{\diffvar}
\newcommand{\remainder}[1]{r_{#1}}
\newcommand{\backback}{-0.15em}
\newcommand{\zeroDimension}[1]{d\hspace{\backback}\left(#1\right)}
\newcommand{\infiniteDimension}[1]{e\hspace{\backback}\left(#1\right)}
\newcommand{\finiteNonZeroDimension}[1]{f\hspace{\backback}\left(#1\right)}
\newcommand{\permutationMatrix}{\mathbf{Q}}
\newcommand{\nodePermutationMatrix}[1]{\permutationMatrix_{#1}}
%\newcommand{\transpose}{^{'}}
\newcommand{\missing}[2]{\delta_{#1 #2}}
\newcommand{\traitIndex}{p}
\newcommand{\arbitraryMatrix}{\mathbf{Z}}
\newcommand{\identityMatrix}[1]{\mathbf{I}_{#1}}
\newcommand{\bigzero}[2]{\makebox(0,0){\hspace{#1}\vspace{#2}\text{\huge0}}}
\newcommand{\altDet}[1]{\hat{\text{det}}\left( #1 \right)}
\newcommand{\changeDimension}[1]{\Delta_{#1}}
\newcommand{\notMissingMatrix}[1]{\boldsymbol{\delta}_{#1}}

\newcommand{\nodePreMean}[1]{\mathbf{n}_{#1}}
\newcommand{\nodePreVariance}[1]{\mathbf{W}_{#1}}
\newcommand{\nodePrePrecision}[1]{\mathbf{Q}_{#1}}
\newcommand{\deflatedNodePrePrecision}[1]{\nodePrePrecision{#1}^{\star}}
%\newcommand{\MVN}{\text{MVN}}
\newcommand{\Indexobs}{\mathcal{I}^{\text{\tiny obs}}_ \nodeIndexOne}
\newcommand{\Indexmis}{\mathcal{I}^{\text{\tiny mis}}_ \nodeIndexOne}

\newcommand{\normalDensityFunction}[3]{{\cal N}\hspace{-0.25em}\left( #1; #2, #3 \right)}
\newcommand{\conditionalMean}[1]{\boldsymbol{\mu}_{#1}}
\newcommand{\dd}{\text{d}}
\def\rowvarresidMissing{\rowvarresid_{{mis}_\nodeIndexOne}}

\newcommand{\symbolicObservedVariance}{0}
\newcommand{\symbolicMissingVariance}{\infty}
\newcommand{\symbolicFiniteVariance}{+}
\newcommand{\partialDiffusionVariance}[2]{\diffusionVariance_{#1 #2}}

\newcommand{\observedVariance}{\partialDiffusionVariance{\symbolicObservedVariance}{\symbolicObservedVariance}}
\newcommand{\missingVariance}{\partialDiffusionVariance{\symbolicMissingVariance}{\symbolicMissingVariance}}
\newcommand{\finiteVariance}{\partialDiffusionVariance{\symbolicFiniteVariance}{\symbolicFiniteVariance}}
\newcommand{\tipLatentPrecision}[1]{\mathbf{R}_{#1}}

\newcommand{\finiteObservedVariance}{\partialDiffusionVariance{\symbolicFiniteVariance}{\symbolicObservedVariance}}
\newcommand{\finiteMissingVariance}{\partialDiffusionVariance{\symbolicFiniteVariance}{\symbolicMissingVariance}}
\newcommand{\observedMissingVariance}{\partialDiffusionVariance{\symbolicObservedVariance}{\symbolicMissingVariance}}
\newcommand{\diag}[1]{\text{diag} \left[ #1 \right]}

\newcommand{\deflatedPrecision}[1]{\mathbf{P}^{\star}_{#1}}
\newcommand{\startingPrecision}[1]{\mathbf{P}_{#1}}
\newcommand{\finitePrecision}[1]{\tilde{\startingPrecision{#1}}}
\newcommand{\bogusPrecision}[1]{\finitePrecision{#1}^{-1}}
\newcommand{\raisedInfty}{\raisebox{0.1em}{$\infty$}}
\newcommand{\bfZero}{\mathbf{0}}
\newcommand{\missingIndicator}[1]{\mathbf{M}_{#1}}

\newcommand{\phylogeny}{\mathscr{F}}
\newcommand{\phylogenyVariance}{\mathbf{\Upsilon}}
\newcommand{\phylogenyPrecision}{\phylogenyVariance\inverse}
\newcommand{\traitPrecision}{\mathbf{P}}

\newcommand{\grandVariance}{\mathbf{\Sigma}}
\newcommand{\grandVarianceExpression}{\traitPrecision\inverse \kronecker \phylogenyVariance}
\newcommand{\grandPrecision}{\grandVariance\inverse}
\newcommand{\grandPrecisionExpression}{\traitPrecision \kronecker \phylogenyPrecision}

\newcommand{\latentData}{\mathbf{Y}}
\newcommand{\vecLatentData}{\mathbf{Z}}
\newcommand{\nTaxa}{N}
\newcommand{\nTraits}{P}

\newcommand{\mvnDensity}[2]{\text{MVN}\hspace{-0.1em}\left(#1, #2\right)}
\newcommand{\mvn}[3]{\text{MVN}\hspace{-0.1em}\left(#1; #2, #3\right)}
\newcommand{\gradient}[1]{\frac{\partial}{\partial #1}}

\newcommand{\rate}[1]{\phi_{#1}}
\newcommand{\allRates}{\boldsymbol{\phi}}
\newcommand{\rank}{\mbox{rank}}
\newcommand{\sVar}{\rm Var}

\newcommand{\distas}[1]{\overset{#1}{\sim}}%

\usepackage{empheq}
\newcommand*\widefbox[1]{\fbox{\hspace{2em}#1\hspace{2em}}}

\usepackage{etoolbox,environ}
\newtoggle{showProof}
\togglefalse{showProof}
\NewEnviron{pf}
  {\iftoggle{showProof}{\BODY}{}}


\usepackage{scalerel,stackengine}
\stackMath
\newcommand\reallywidehat[1]{%
\savestack{\tmpbox}{\stretchto{%
  \scaleto{%
    \scalerel*[\widthof{\ensuremath{#1}}]{\kern-.6pt\bigwedge\kern-.6pt}%
    {\rule[-\textheight/2]{1ex}{\textheight}}%WIDTH-LIMITED BIG WEDGE
  }{\textheight}% 
}{0.5ex}}%
\stackon[1pt]{#1}{\tmpbox}%
}
\colorlet{shadecolor}{gray!15}

\DeclareMathAlphabet{\mathcal}{OMS}{cmsy}{m}{n}
\newcommand{\columnSpace}[1]{\mathnormal{C}(#1)}
\newcommand{\dimm}{\mbox{dim}}
\newcommand{\Null}[1]{\mathcal{N}(#1)}

\def\treeTraitsRoot{\boldsymbol{\mu}_R}
\def\diffprec{\boldsymbol{\Lambda}_R}
\def\diffprecprior{\boldsymbol{\Lambda}_{R_0}}
\def\diffmean{\boldsymbol{\mu}_0}
\def\pss{\kappa_0}
\def\diffprecpriordf{\nu}
\def\pss{\kappa_0}

\newtheorem{thm}{Theorem}
\newtheorem{lemma}{Lemma}
\newtheorem{prop}{Proposition}
\newtheorem{cor}{Corollary}
\newtheorem{remark}{Remark}
\newtheorem{example}{Example}
\newtheorem{mydef}{Definition}
\newtheorem{assumption}{Assumption}

\usepackage{chngcntr}
\counterwithin{figure}{section}

\DeclareMathOperator*{\argmax}{arg\,max}
\DeclareMathOperator*{\argmin}{arg\,min}
