\setcounter{section}{27}

\section{Lecture 28: April 5}


\subsection*{Last time}
\begin{itemize}
	\item two-way ANOVA
\end{itemize}


\subsection*{Today}
\begin{itemize}
	\item Announcement: per requested by three students, we will do a third poll on Wednesday for the alternative grading path (the last time).
	\item Lab session review
	\item ANCOVA
	\item Linear contrasts of means
\end{itemize}

\subsubsection*{Additional reference}
\href{https://www4.stat.ncsu.edu/~osborne/st512r/handouts/allpackets.pdf}{Course notes} by Dr. Jason Osborne.

\subsubsection*{A three-factor example}
In a balanced, complete, crossed design, $N=36$ shrimp were randomized to $abc = 12$ treatment combinations from the factors below:
\begin{itemize}
	\item A1: Temperature at $25^{\circ}C$
	\item A2: Temperature at $35^{\circ}C$
	\item B1: Density of shrimp population at $80$ shrimp/40$l$
	\item B2: Density of shrimp population at $160$ shrimp/40$l$
	\item C1: Salinity at $10$ units
	\item C2: Salinity at $25$ units
	\item C3: Salinity at $40$ units
\end{itemize}
The response variable of interest is weight gain $Y_{ijkl}$ after four weeks.

\subsubsection*{Three-way ANOVA model}
$$
\begin{aligned}
Y_{ijkl} &= \mu + \alpha_i + \beta_j + \gamma_k\\
&+ (\alpha \beta)_{ij} + (\alpha \gamma)_{ik} + (\beta \gamma)_{jk}\\
&+ (\alpha\beta \gamma)_{ijk} + \epsilon_{ijkl}\\
\end{aligned}
$$
$$
\begin{aligned}
	i &= 1, 2\\
	j &= 1, 2\\
	k &= 1, 2, 3\\
	l &= 1, 2, 3\\
	\epsilon_{ijkl} &\distas{iid} \mathcal{N}(0, \sigma^2)\\	
\end{aligned}
$$
Many constraints such as (over one dimension):
$$
\begin{aligned}
	\sum\limits_i \alpha_i &= 0\\
	\sum\limits_i (\alpha \beta)_{ij} &= \sum\limits_j (\alpha \beta)_{ij} = 0 \quad \mbox{ for all } i, j\\
	\sum\limits_i (\alpha \beta \gamma)_{ijk} &=  \sum\limits_j (\alpha \beta \gamma)_{ijk} = \sum\limits_k (\alpha \beta \gamma)_{ijk} = 0 \quad \mbox{ for all }i, j, k\\
\end{aligned}
$$

Now, please finish the table below
\begin{table}[H]
	\renewcommand{\arraystretch}{1.5}
	\centering
	\begin{tabular}{lc}
		\toprule
		Source & df\\
		\hline
		A & \\
		B & \\
		C & \\
		$A \times B$ & \\
		$A \times C$ & \\
		$B \times C$ & \\
		$A \times B \times C$ &\\
		Residual & \\
		\hline
		Total &\\
		\bottomrule
	\end{tabular}
\end{table}
{\it Answer: }\\
\begin{pf}
	\begin{table}[H]
		\renewcommand{\arraystretch}{1.5}
		\centering
	\begin{tabular}{lc}
	\toprule
	Source & df\\
	\hline
	A & 1\\
	B & 1\\
	C & 2\\
	$A \times B$ & 1\\
	$A \times C$ & 2\\
	$B \times C$ & 2\\
	$A \times B \times C$ & 2\\
	Residual & 24\\
	\hline
	Total & 35\\
	\bottomrule
\end{tabular}
	\end{table}		
\end{pf}

The three-way ANOVA model includes parameters for
\begin{itemize}
	\item Main effects: $\alpha_i$, $\beta_j$ and $\gamma_k$.
	\item Two-way interactions between each pair of factors: $(\alpha \beta)_{ij}$, $(\alpha \gamma)_{ik}$ and $(\beta \gamma)_{jk}$.
	\item Three-way interaction among all three factors: $(\alpha \beta \gamma)_{ijk}$.
\end{itemize}

Readings:
\begin{enumerate}
	\item JF 8.3.1 on parameter estimates and hypothesis testing for three-way ANOVA model.
	\item JF 8.3.2 on Higher-order classifications.
\end{enumerate}


\subsection*{Analysis of Covariance}
\underline{Analysis of covariance} (ANCOVA) is a term used to describe linear models that contain both qualitative and quantitative explanatory variables.
The method is, therefore, equivalent to dummy-variable regression, discussed in the previous lectures, although the ANCOVA model is parametrized differently from the dummy-regression model.

\underline{Covariate} is a variable known to affect the response that
\begin{enumerate}
	\item differs among EUs
	\item reflects differences that exist independently of experimental treatment.
\end{enumerate}

\subsubsection*{A nutrition example}
A nutrition scientist conducted an experiment to evaluate the effects of four vitamin supplements on the weight gain of laboratory animals.
The experiment was conducted in a completely randomized design with $N=20$ animals randomized to $a=4$ supplement groups, 
each with sample size $n \equiv 5$.
The response variable of interest is weight gain, but calorie intake $z$ was measured simultaneously.
\begin{table}[H]
	\renewcommand{\arraystretch}{1.5}
	\centering
	\begin{tabular}{|cc|cc|cc|cc|}
		\toprule
		Diet & $y(g)$ & Diet & $y$& Diet & $y$& Diet & $y$\\
		\hline
		1 & 48 & 2 & 65 & 3 & 79 & 4 &59 \\
		1 & 67 & 2 & 49 & 3 & 52 & 4 &50 \\
		1 & 78 & 2 & 37 & 3 & 63 & 4 &59 \\
		1 & 69 & 2 & 75 & 3 & 65 & 4 &42 \\
		1 & 53 & 2 & 63 & 3 & 67 & 4 &34 \\
		\hline
		1 & $\bar{y}_{1+} = 63$ & 2 & $\bar{y}_{2+} = 57.8$ & 3 & $\bar{y}_{3+} = 65.2$ & 4 &$\bar{y}_{4+} = 48.8$ \\
		1 & $s_1 = 12.3$ & 2 & $s_2 = 14.9$ & 3 & $s_3 = 9.7$ & 4 &$s_4 = 10.9$ \\
		\bottomrule
	\end{tabular}
\end{table}

Question: Is there evidence of a vitamin supplement effect?

\begin{table}[H]
	\renewcommand{\arraystretch}{1.5}
	\centering
	\begin{tabular}{lrrrrr}
		& Df & Sum Sq & Mean Sq & F value & Pr($>$F)\\
	Diet & 3 & 	797.8  & 265.9 &  1.823 & 0.184\\
	Residuals  & 16& 2334.4 &  145.9 & &\\              	
	\end{tabular}
\end{table}
\begin{pf}
Conclusion: at $\alpha = 0.05$ level, there is no significant difference between vitamin supplement levels on weight gain.
\end{pf}

But calorie intake $z$ was measured simultaneously:
\begin{table}[H]
	\renewcommand{\arraystretch}{1.5}
	\centering
	\begin{tabular}{ccc|ccc|ccc|ccc}
		\toprule
		Diet & $y(g)$ & $z$ & Diet & $y$ & $z$& Diet & $y$ & $z$& Diet & $y$ & $z$\\
		\hline
		1 & 48 & 350 & 2 & 65 & 400 & 3 & 79 & 510 & 4 & 59 & 530\\
		1 & 67 & 440 & 2 & 49 & 450 & 3 & 52 & 410 & 4 & 50 & 520\\
		1 & 78 & 440 & 2 & 37 & 370 & 3 & 63 & 470 & 4 & 59 & 520\\
		1 & 69 & 510 & 2 & 75 & 530 & 3 & 65 & 470 & 4 & 42 & 510\\
		1 & 53 & 470 & 2 & 63 & 420 & 3 & 67 & 480 & 4 & 34 & 430\\						
		\bottomrule
	\end{tabular}
\end{table}

Question: How and why could these new data be incorporated into analysis?\\
Answer: ANCOVA can be used to reduce unexplained variation.

ANCOVA model, 
$$
y_{ij} = \mu + \alpha_i + \beta z_{ij} + \epsilon_{ij}
$$
where $\mu$ is the reference level, $\alpha_i$ is the main effect of treatment, $\beta$ is the partial regression coefficient, 
and $\epsilon_{ij} \distas{iid} \mathcal{N}(0, \sigma^2)$.
The model is equivalent as the dummy-variable regression model,
$$
Y_i = \beta_0 + \beta_1 x_{i1} + \beta_2 x_{i2} + \beta_3 x_{i3} + \beta_z z_i + \epsilon_i \quad \mbox{ for } i = 1, \dots, 20
$$

Finish the table below
\begin{table}[H]
	\renewcommand{\arraystretch}{1.5}
	\centering
	\begin{tabular}{lr}
		\toprule
		Source & df\\
		\hline
		Diet & \\
		Covariate & 1\\
		Residual & \\
		\hline
		Total &\\
		\bottomrule
	\end{tabular}
\end{table}
{\it Answer: }\\
\begin{pf}
\begin{table}[H]
	\renewcommand{\arraystretch}{1.5}
	\centering
	\begin{tabular}{lr}
		\toprule
		Source & df\\
		\hline
		Diet & 3\\
		Covariate & 1\\
		Residual & 15\\
		\hline
		Total & 19\\
		\bottomrule
	\end{tabular}
\end{table}
\end{pf}

To test for difference among treatments.
The null hypothesis in terms of $\alpha_i$ is\\
$H_0: \alpha_1 = \alpha_2  = \dots = \alpha_4 = 0$ v.s. $H_a: \mbox{ at least one } \alpha_i \ne 0$\\
And the null hypothesis in terms of $\beta_i$ is\\
$H_0: \beta_1 = \beta_2  = \beta_3 = 0$ v.s. $H_a: \mbox{ at least one } \beta_i \ne 0$

Question: which two models do we compare when testing the above null hypothesis?
{\it Answer: }\\
\begin{pf}
\begin{itemize}
	\item In terms of ANOVA and ANCOVA models, we compare the one-way ``ANOVA'' model (actually the simple linear regression model) with only the covariate term to the ANCOVA model that has both the covariate and the treatment.\\
	\colorbox{shadecolor}{aov(y $\sim$ z, data = vitamin.supplement)} vs \colorbox{shadecolor}{aov(y $\sim$ Diet + z, data = vitamin.supplement)} 
	\item In terms of the dummy-variable regression model, we compare the simple linear regression model of regression $y$ on $z$ to the model that includes the dummy-variable for Diet (treatments).\\
	\colorbox{shadecolor}{lm(y $\sim$ z, data = vitamin.supplement)} vs \colorbox{shadecolor}{lm(y $\sim$ Diet + z, data = vitamin.supplement)} 
\end{itemize}
\end{pf}

\subsection*{Linear contrasts of means}
With ANOVA (or ANCOVA) models, we do not generally test hypotheses about individual coefficients (but we can do so if we wish).
For dummy-coded regressors in one-way ANOVA, a $t$-test or $F$-test of $H_0: \alpha_1 = 0$, for example, is equivalent to testing for the difference in means between the first group and the baseline group, $H_0: \mu_1 = \mu_m$.

Consider the one-way ANOVA model:
$$
Y_{ij} = \mu_i + \epsilon_{ij}, i = 1, 2, \dots, t, \mbox{ and } j = 1, 2, \dots, n_i
$$
with $\epsilon_{ij} \distas{iid} \mathcal{N}(0, \sigma^2)$.

A linear function of the group means of the form
$$
\theta = c_1 \mu_1 + c_2 \mu_2 + \dots + c_t \mu_t
$$
is called a \underline{linear combination} of the treatment means.
And the $c_i$'s are the \underline{coefficients} of the linear combination.
If
$$
c_1 + c_2 + \dots + c_t = \sum\limits_{j = 1}^t c_j = 0,
$$
the linear combination is called a \underline{contrast}.
Contrasts with more than two non-zero coefficients are called \underline{complex contrasts}.

Let two contrasts $\theta_1$ and $\theta_2$ be given by
$$
\begin{aligned}
	\theta_1 &= c_1 \mu_1 + \dots + c_t \mu_t = \sum\limits_{j = 1}^t c_j \mu_j\\
	\theta_2 &= d_1 \mu_1 + \dots + d_t \mu_t = \sum\limits_{j = 1}^t d_j \mu_j,\\	
\end{aligned}
$$
then the two contrasts $\theta_1$ and $\theta_2$ are \underline{mutually orthogonal} if the products of their coefficients sum to zero:
$$
c_1 d_1 + \dots + c_t d_t = \sum\limits_{j=1}^t c_j d_j = 0
$$

$\theta_i$ and $\theta_j$ are orthogonal $\implies$ $\hat{\theta}_i$ and $\hat{\theta}_j$ are statistically independent.

\subsubsection*{Types of effects}
Consider the following two-way ANOVA model:
$$
\begin{aligned}
Y_{ijk} &= \mu + \alpha_i + \beta_j + (\alpha \beta)_{ij} + \epsilon_{ijk}\\
i &= 1, 2 = a \mbox{ and } j = 1, 2=b \mbox{ and } k = 1, 2, \dots , 7 = n.\\
\end{aligned}
$$
$\epsilon_{ijk} \distas{iid} \mathcal{N}(0, \sigma^2)$.  Parameter constraints: $\sum_i \alpha_i = \sum_j \beta_j = 0$ and 
$\sum_i (\alpha \beta)_{ij} = 0$ for each $j$ and $\sum_j (\alpha \beta)_{ij} = 0$ for each $i$.
\begin{itemize}
	\item Factor A: AGE has $a = 2$ levels - $A_1: $ younger and $A_2: $ older
	\item Factor B: GENDER has $b = 2$ levels - $B_1: $ female and $B_2: $ male
\end{itemize}

Three kinds of effects in this $2 \times 2$ design:
\begin{enumerate}
	\item \underline{Simple effects} are simple contrasts.
	    \begin{itemize}
	    	\item $\mu(A_1 B) = \mu_{12} - \mu_{11}$ - simple effect of gender for young folks.
	    	\item $\mu(A B_1) = \mu_{21} - \mu_{11}$ - simple effect of age for women. 
	    \end{itemize}
    \item \underline{Interaction effects} are differences of simple effects:
    $\mu(AB) = \mu(AB_2) - \mu(AB_1) = (\mu_{22} - \mu_{12}) - (\mu_{21} - \mu_{11})$
    \begin{itemize}
    	\item difference between simple age effects for men and women
    	\item difference between simple gender effects for old and young folks
    	\item interaction effect of AGE and GENDER.
    \end{itemize}
    \item \underline{Main effects} are averages or sums of simple effects
    $$
    \begin{aligned}
    	\mu(A) &= \frac{1}{2} (\mu(AB_1) + \mu(AB_2))\\
    	\mu(B) &= \frac{1}{2} (\mu(A_1 B) + \mu(A_2B))\\    	
    \end{aligned}
    $$
\end{enumerate}







